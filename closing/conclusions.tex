Luego de haber finalizado el proyecto, se concluye lo siguiente:

\begin{itemize}
	
	\item El uso de la tecnología en el comercio y las estrategias de ventas como “fiar”, son unos pilares influyentes en el aumento de las ventas, pues estos permiten a sus consumidores acceder de forma fácil a sus productos y pagarlos cómodamente, generando satisfacción y fidelización en el cliente.
	
	\item Exponer de forma detallada y en tiempo real, las compras, deudas o transacciones realizadas entre el tendero y consumidor, genera un alto grado de confianza al usuario, ya que este, no estará con la incertidumbre referente a los pagos a realizar o ya realizados.
	
	\item Mostrar el monto total de las deudas y prestamos de los usuarios, es un mecanismo que ayuda autocontrolar a los mismos en cuanto a la adquisición de nuevas deudas se refiere, pues ellos sabrán lo máximo que pueden endeudarse y con base a eso, limitarse, de igual forma, ayuda a controlar el monto en prestamos a realizar, pues el usuario será consciente de lo máximo que le podrá prestar a otro.
	
\end{itemize}